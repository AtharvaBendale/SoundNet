\documentclass[11pt]{article}

\usepackage{amsmath}
\usepackage{amssymb}
\usepackage{hyperref}
\usepackage{ulem}
\usepackage{graphicx}
\usepackage{subcaption}
\usepackage{gensymb}
\usepackage{tcolorbox}
\usepackage[margin=0.5in]{geometry}
\usepackage{tikz}
\usepackage{listings}
\usepackage{xcolor}
\usepackage{fancyvrb}

\usepackage[utf8]{inputenc}
\usepackage{pmboxdraw}

\lstdefinestyle{folderstyle}{
  basicstyle=\ttfamily,
  backgroundcolor=\color{gray!10},
  frame=single,
  columns=flexible,
  keepspaces=true,
  breaklines=true
}

\usetikzlibrary{shapes.geometric, arrows}

\tikzstyle{startstop} = [ellipse, minimum width=1.5cm, minimum height=1cm, text centered, draw=black, fill=gray!30]
\tikzstyle{process} = [rectangle, minimum width=1.5cm, minimum height=1cm, text centered, draw=black, fill=orange!30]
\tikzstyle{decision} = [diamond, minimum width=1.5cm, minimum height=1cm, text centered, draw=black, fill=green!30]
\tikzstyle{arrow} = [thick,->,>=stealth]

% Define the tcolorbox style for single line boxes without titles
\tcbuselibrary{listings}
\tcbset{
  mybox/.style={
    colframe=black,        % Frame color
    colback=black!25!white, % Background color
    boxrule=0.5mm,          % Thickness of the frame
    arc=1mm,                % Rounded corners
    outer arc=1mm,
    boxsep=2mm,             % Space between the frame and the content
    height=5ex,            % Height of the box
    width=0.9\textwidth,    % Width of the box
    top=0pt,                % No extra space on top
    bottom=0pt,             % No extra space on bottom
    left=1mm,               % Space on the left
    right=1mm               % Space on the right
  }
}

\title{Program Documentation: CS 378 Lab 2}
\author{Atharva Bendale (22B0901), Nivesh Aggarwal (22B0912), \\ Dhvanil Gheewala (22B0923), Vishal Bysani (22B1061)}

\begin{document}
% \setlength{\itemsep}{-1cm}
\maketitle

\section{Included Files and Dependencies}

\subsection*{Dependencies}
\begin{itemize}
    \item NumPy:
    \begin{tcolorbox}
      \texttt{pip install numpy}
    \end{tcolorbox}
    
    \item SciPy:
    \begin{tcolorbox}
      \texttt{pip install scipy}
    \end{tcolorbox}
    
    \item Pyaudio:
    \begin{tcolorbox}
      \texttt{sudo apt-get install portaudio19-dev}\\
    % \end{tcolorbox}
    % \begin{tcolorbox}[mybox]
        \texttt{pip install pyaudio}
    \end{tcolorbox}
\end{itemize}

\subsection*{Files included}
\begin{Verbatim}[frame=single]
22b0901_22b0912_22b0923_22b01061_CS378_lab2.tar.gz/
├── main.py
├── receiver.py
├── sender.py
├── crc.py
└── Report/
    ├── 22b0901_22b0912_22b0923_22b01061_dd_CS378_lab2.pdf
    ├── 22b0901_22b0912_22b0923_22b01061_dd_CS378_lab2.tex
    ├── documentation.pdf
    └── documentation.tex
\end{Verbatim}

% \noindent \textbf{main.py:} Main script for sending and receiving binary messages using CRC encoding, audio transmission, and error correction.\\
% \noindent \textbf{receiver.py:} Receives and decodes audio signals into binary messages using Welch's method of frequency analysis.\\
% \noindent \textbf{sender.py:} Generates and transmits audio signals representing binary messages using frequency modulation.\\
% \noindent \textbf{crc.py:} Implements 2-bit error correction with CRC generating polynomials with length depending upon the length of the input message.
\subsubsection*{main.py}
Main script for sending and receiving binary messages using CRC encoding, audio transmission, and error correction\\
\textbf{Functions:}\vspace{-0.25cm}
\begin{itemize}\setlength\itemsep{-0.1cm}
    \item \textbf{send :} Takes input message, number of errors and its position from the user , creates transmission message by appending special sequence, preamble and CRC bits
    \item \textbf{recv :} Calibrates the system for background noise, receives signals, decodes it to bits and corrects the error bits and prints the message
\end{itemize}
\subsubsection*{receiver.py}
Receives and decodes audio signals into binary messages using Welch's method of frequency analysis\\
\textbf{Functions:}\vspace{-0.25cm}
\begin{itemize}\setlength\itemsep{-0.1cm}
    \item \textbf{open\_audio\_stream :} Opens the audio stream
    \item \textbf{receive\_audio :} Receive an audio signal from the default audio input device for the specified duration
    \item \textbf{calibrate :} Calculates the ambient noise power for each frequency range
    \item \textbf{decode\_audio\_to\_bits :} Decode an audio signal to a list of bits
\end{itemize}
\subsubsection*{sender.py}
Generates and transmits audio signals representing binary messages using frequency modulation\\
\textbf{Functions:}\vspace{-0.25cm}
\begin{itemize}\setlength\itemsep{-0.1cm}
    \item \textbf{generate\_tone :} Generate a sine wave tone of given frequency and duration
    \item \textbf{encode\_bits\_to\_audio :} Encode a list of bits into an audio signal
    \item \textbf{send\_audio :} Send an audio signal to the default audio output device
\end{itemize}
\subsubsection*{crc.py}
Implements 2-bit error correction with CRC generating polynomials with length depending upon the length of the input message\\
\textbf{Functions:}\vspace{-0.25cm}
\begin{itemize}\setlength\itemsep{-0.1cm}
    \item \textbf{preamble :} returns the binary representation of preamble
    \item \textbf{polyDivision :} Performs polynomial division for CRC calculation
    \item \textbf{bruteCheck :} Iterates over all possible double/triple bit errors and checks whether the modified dividend is perfectly divisible by the polynomial for finding the correct message
    \item \textbf{encodeCrc :} Encodes the given message and adds redundancy using CRC for error detection and correction
    \item \textbf{decodeCrc :} Decodes the given transmission, detects and corrects errors
\end{itemize}
\section{Usage}

\subsection*{Sender's End}

To initiate the transmission process, follow these steps:

\begin{tcolorbox}
    python3 main.py - \hspace{-0.1cm}-send
\end{tcolorbox}

\begin{enumerate}
    \item Enter the message you wish to transmit.
    \item Specify the number of bit errors you want to introduce.
    \item Provide the fractions at which you want to introduce the errors.
    \item Press Enter to start the transmission.
\end{enumerate}

\subsection*{Receiver's End}

To receive and process the transmitted message, follow these steps:

\begin{tcolorbox}
    python3 main.py - \hspace{-0.1cm}-recv
\end{tcolorbox}

\begin{enumerate}
    \item Run the command above in the terminal.
    \item Wait for the incoming transmission.
    \item The received message will be displayed in the terminal.
\end{enumerate}

\end{document}
